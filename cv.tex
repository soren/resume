% TODO: add links to linkedin, twitter, etc
% TODO: add links to projects, on cpan and github

\documentclass[a4paper,11pt]{article}

\topmargin=0.0in
\oddsidemargin=0.0in
\evensidemargin=0in
\textwidth=6.5in
\marginparwidth=0.5in
\headheight=0pt
\headsep=0pt
\textheight=10.0in

\usepackage[utf8]{inputenc}
\usepackage{parskip}
\usepackage{marvosym}
\usepackage{hyperref}

\newcommand\redacted{[skjult i online version]}

\newcommand\myaddressone{\redacted}
\newcommand\myaddresstwo{\redacted}
\newcommand\myphone{\redacted}

%
% By default address and phonenumber are redacted. To display address
% and phonenumber create the file myinfo.tex which should override the
% three commands: \myaddressone, \myaddresstwo and \myphone, e.g:
%
%    \renewcommand\myaddressone{Street 123}
%    \newcommand\myaddresstwo{City Zip}
%    \newcommand\myphone{5555 5555}
%

\include{myinfo}

\begin{document}

\centerline{\fontfamily{phv}\selectfont{\Huge\textbf{CV}}}
\bigskip

\begin{minipage}[b]{0.30\textwidth}
  \fontfamily{phv}\selectfont%
  \textbf{Søren Lund} \\
  \myaddressone \\
  \myaddresstwo \\
  \myphone \\
  \Email\ soren@lund.org
\end{minipage}%
\hfill
\begin{minipage}[b]{0.30\textwidth}
  \fontfamily{phv}\selectfont%
  Jeg er en alsidig it-konsulent med
  mere end 20 års professionel
  erfaring. Jeg har en bred vifte af
  kompetencer, og
  kan lide nye udfordringer.
\end{minipage}

\bigskip
\centerline{\small\textit{Nyeste udgave kan hentes her: \url{http://slu.sdf.org/resume/cv.pdf}}}
\centerline{\small\textit{Dato og tid for seneste ændring af denne udgave: % TODO: add links to linkedin, twitter, etc
% TODO: add links to projects, on cpan and github

\documentclass[a4paper,11pt]{article}

\topmargin=0.0in
\oddsidemargin=0.0in
\evensidemargin=0in
\textwidth=6.5in
\marginparwidth=0.5in
\headheight=0pt
\headsep=0pt
\textheight=10.0in

\usepackage[utf8]{inputenc}
\usepackage{parskip}
\usepackage{marvosym}

\newcommand\redacted{[skjult i online version]}

\newcommand\myaddressone{\redacted}
\newcommand\myaddresstwo{\redacted}
\newcommand\myphone{\redacted}

%
% By default address and phonenumber are redacted. To display address
% and phonenumber create the file myinfo.tex which should override the
% three commands: \myaddressone, \myaddresstwo and \myphone, e.g:
%
%    \renewcommand\myaddressone{Street 123}
%    \newcommand\myaddresstwo{City Zip}
%    \newcommand\myphone{5555 5555}
%

\include{myinfo}

\begin{document}

\centerline{\fontfamily{phv}\selectfont{\Huge\textbf{CV}}}
\bigskip

\begin{minipage}[b]{0.30\textwidth}
  \fontfamily{phv}\selectfont%
  \textbf{Søren Lund} \\
  \myaddressone \\
  \myaddresstwo \\
  \myphone \\
  \Email\ soren@lund.org
\end{minipage}%
\hfill
\begin{minipage}[b]{0.30\textwidth}
  \fontfamily{phv}\selectfont%
  Jeg er en alsidig it-konsulent med
  mere end 20 års professionel
  erfaring. Jeg har en bred vifte af
  kompetencer, og
  kan lide nye udfordringer.
\end{minipage}

\bigskip
\centerline{\small\textit{Nyeste udgave kan hentes her: http://slu.sdf.org/resume/cv.pdf}}
\centerline{\small\textit{Dato og tid for seneste ændring af denne udgave: % TODO: add links to linkedin, twitter, etc
% TODO: add links to projects, on cpan and github

\documentclass[a4paper,11pt]{article}

\topmargin=0.0in
\oddsidemargin=0.0in
\evensidemargin=0in
\textwidth=6.5in
\marginparwidth=0.5in
\headheight=0pt
\headsep=0pt
\textheight=10.0in

\usepackage[utf8]{inputenc}
\usepackage{parskip}
\usepackage{marvosym}

\newcommand\redacted{[skjult i online version]}

\newcommand\myaddressone{\redacted}
\newcommand\myaddresstwo{\redacted}
\newcommand\myphone{\redacted}

%
% By default address and phonenumber are redacted. To display address
% and phonenumber create the file myinfo.tex which should override the
% three commands: \myaddressone, \myaddresstwo and \myphone, e.g:
%
%    \renewcommand\myaddressone{Street 123}
%    \newcommand\myaddresstwo{City Zip}
%    \newcommand\myphone{5555 5555}
%

\include{myinfo}

\begin{document}

\centerline{\fontfamily{phv}\selectfont{\Huge\textbf{CV}}}
\bigskip

\begin{minipage}[b]{0.30\textwidth}
  \fontfamily{phv}\selectfont%
  \textbf{Søren Lund} \\
  \myaddressone \\
  \myaddresstwo \\
  \myphone \\
  \Email\ soren@lund.org
\end{minipage}%
\hfill
\begin{minipage}[b]{0.30\textwidth}
  \fontfamily{phv}\selectfont%
  Jeg er en alsidig it-konsulent med
  mere end 20 års professionel
  erfaring. Jeg har en bred vifte af
  kompetencer, og
  kan lide nye udfordringer.
\end{minipage}

\bigskip
\centerline{\small\textit{Nyeste udgave kan hentes her: http://slu.sdf.org/resume/cv.pdf}}
\centerline{\small\textit{Dato og tid for seneste ændring af denne udgave: % TODO: add links to linkedin, twitter, etc
% TODO: add links to projects, on cpan and github

\documentclass[a4paper,11pt]{article}

\topmargin=0.0in
\oddsidemargin=0.0in
\evensidemargin=0in
\textwidth=6.5in
\marginparwidth=0.5in
\headheight=0pt
\headsep=0pt
\textheight=10.0in

\usepackage[utf8]{inputenc}
\usepackage{parskip}
\usepackage{marvosym}

\newcommand\redacted{[skjult i online version]}

\newcommand\myaddressone{\redacted}
\newcommand\myaddresstwo{\redacted}
\newcommand\myphone{\redacted}

%
% By default address and phonenumber are redacted. To display address
% and phonenumber create the file myinfo.tex which should override the
% three commands: \myaddressone, \myaddresstwo and \myphone, e.g:
%
%    \renewcommand\myaddressone{Street 123}
%    \newcommand\myaddresstwo{City Zip}
%    \newcommand\myphone{5555 5555}
%

\include{myinfo}

\begin{document}

\centerline{\fontfamily{phv}\selectfont{\Huge\textbf{CV}}}
\bigskip

\begin{minipage}[b]{0.30\textwidth}
  \fontfamily{phv}\selectfont%
  \textbf{Søren Lund} \\
  \myaddressone \\
  \myaddresstwo \\
  \myphone \\
  \Email\ soren@lund.org
\end{minipage}%
\hfill
\begin{minipage}[b]{0.30\textwidth}
  \fontfamily{phv}\selectfont%
  Jeg er en alsidig it-konsulent med
  mere end 20 års professionel
  erfaring. Jeg har en bred vifte af
  kompetencer, og
  kan lide nye udfordringer.
\end{minipage}

\bigskip
\centerline{\small\textit{Nyeste udgave kan hentes her: http://slu.sdf.org/resume/cv.pdf}}
\centerline{\small\textit{Dato og tid for seneste ændring af denne udgave: \input{cv.tex.ts}}}

\section*{Tidligere og nuværende jobs}

\textbf{AWS Solution Architect} \hfill 2018 -- (nu) \\
\textsl{KeyCore P/S}

Jeg har arbejdet som konsulent for kunder, der benytter eller har
tænkt sig at benytte Amazone Web Services (AWS).

\smallskip

\textbf{Freelancekonsulent} \hfill 2012 -- 2018 \\
\textsl{369 Consult}

\smallskip

\textbf{Seniorkonsulent, ekstern} \hfill 2014 -- (nu) \\
\textsl{Nordea A/S}

Jeg har været med til udvikling, fejlretning og 3rd level support på
en moderne webløsning til blandt andet valutahandel. Løsningen er
baseret på RESTful micro services, Dropwizard, Oracle DB, Lucene,
Hibernate og AngularJS. Desuden har jeg stået for implementeringen af
automatiseret byg, test og udrulning i Jenkins og Bamboo.

Webløsningen afløser et tidligere produkt baseret på Java Swing, Java
Web Start, Oracle DB og med flere integrationer til bagvedliggende
systemer. Jeg har også hjulpet med videreudvikling, fejlretning og 3rd
level support af dette produkt.

\smallskip

\textbf{Seniorkonsulent, ekstern} \hfill 2012 -- 2013 \\
\textsl{Visma Consulting A/S}

Fortsat som konsulent hos Visma Consulting, nu som selvstændig, se
nedenfor.

\smallskip

\textbf{Seniorkonsulent} \hfill 2011 -- 2012 \\
\textsl{Solfisk Konsulenter Aps}

Arbejdet som ekstern konsulent hos Visma Consulting, hvor jeg har
været med til at udvikle to webløsninger baseret på bl.a. JSF 2, Seam,
JBoss AS, Hibernate, XSL-FO (PDF), Web Services og Oracle
DB. Udviklingen har været styret vha.\ Scrum og jeg har arbejdet med
alle dele af udviklingen, fra back-end til front-end.  Har desuden
arbejdet med AM og videreudvikling på både disse systemer og andre
systemer, der vedligeholdes af Visma Consulting.

\smallskip

\textbf{Senior Systemudvikler} \hfill 2010 -- 2011 \\
\textsl{LeasingBørsen Aps (nu GoLease.com)}

Jeg var en del af udviklingsafdelingen, men jeg havde også et ben
driftsafdelingen. Deltog i videreudvikling og vedligehold af
forskellige egenudviklede systemer, først og fremmest en
web-applikation, der var skrevet i Perl, Catalyst og CouchDB.  Al
udvikling var styret vha.\ Scrum. Jeg var derudover, som Release
Manager, ansvarlig for integrationstest og udrulning i både test og
produktion.

\smallskip

\textbf{Seniorkonsulent} \hfill 2006 -- 2010 \\
\textsl{Sirius IT A/S (opr. TietoEnator A/S)}

Medlem af et team med fokus på én kunde og dennes webbaserede
systemer. Der har været tale om både projekter med nyudvikling samt
support og fejlretning på de eksisterende systemer. Min rolle har
primært været teknisk funderet og indadvendt mod teamet, først og
fremmest som Build Master og dermed med det tekniske ansvar for
leverancerne, og dernæst som udvikler og supporter.

\smallskip

\textbf{Senior Systemudvikler} \hfill 2000 -- 2006 \\
\textsl{CSC Consulting Group (opr. eHuset)}

Involveret i kundeprojekter af varierende størrelse. Dette har primært
været webløsninger, oftes med en bagvedliggende database og/eller
integration til eksisterende systemer. Min rolle har varieret en del,
da jeg både har arbejdet med implementering, design, behovsafdækning
(hos/med kunder), støtte for salg, support og løsningsarkitekt.

\smallskip

\textbf{Softwareudvikler} \hfill 1996 -- 2000 \\
\textsl{Dansk Maritimt Institut (nu Force Technology)}

Udvikling og vedligehold af in-house simulatorsystem og
produktificering af samme. Mine opgaver var primært udvikling, på
flere platforme og med flere teknologier – alt fra små scripts, over
biblioteker/moduler til komplette brugergrænseflader.


\section*{Uddannelse}

\textbf{Civilingeniør} \hfill 1987 -- 1995 \\
\textsl{Danmarks Tekniske Universitet}

Mit afgangsspeciale hed \textit{Rent parallaktisk stereo}, og blev
skrevet på Instituttet for matematisk modellering (IMM).


\section*{Efteruddannelse \& kurser}

\textbf{Nyheder i Java EE 6} \hfill (1 dag) 2010 \\
\textsl{Lund\&Bendsen}

\textbf{Introduktion til ITIL} \hfill (1 dag) 2008 \\
\textsl{(Internt kursus h/Sirius IT -- ekstern underviser)}

\textbf{Introduktion PRINCE2} \hfill (1 dag) 2008 \\
\textsl{(Internt kursus h/Sirius IT -- ekstern underviser)}

\textbf{BEA kursus WLP-D11-70-01-CW-2} \hfill (5 dage) 2003 \\
\textsl{Development with BEA WebLogic Portal}

\textbf{BEA selvstudie WLS-D11-61-01-CW} \hfill (2 dage) 2003 \\
\textsl{Fundamentals of J2EE Web Application Development\\WebLogic Server 6.1}

%%\textbf{Skriv bedre} \hfill (1 dag) 2000 \\
%%\textsl{IDA}
%%
%%\textbf{Præsentationsteknik} \hfill (1 dag) 2002 \\
%%\textsl{IDA}

\textbf{Databaser og Internet (ASP, SQL)} \hfill (5 dage) 1999 \\
\textsl{Dieu}

\textbf{OOD med UML notation} \hfill (3 dage) 1998 \\
\textsl{Objektorienteret Design og C++ / Java Implementering\\Teknologisk Institut}


\section*{Kompetencer}

\subsection*{Overordnede kompetencer}

\begin{description}

  \item[Kvalitetssikring/koordinering --] Jeg har arbejdet en del
    med konfigurationsstyring, automatiserede tests og Release
    Management.

  \item[Løsningsarkitekt --] Jeg er generalist, med en bred faglig
    baggrund, og får nemt overblikket over komplicerede systemer.

  \item[Tilbudsgivning/estimering --] Jeg har været med til at
    udarbejde flere, både små og store, tilbud. Desuden har jeg både
    alene og med forretningskonsulenter være med til at skrive
    anbefalinger/udbudsmateriale for kunder.

  \item[Forretningsforståelse --] Jeg er vant til kundekontakt og
    støtter ofte i forbindelse med pre-sales.

  \item[Systemintegration --] Jeg har bred teknisk viden, som jeg
    blandt andet har benyttet til integration af legacy-systemer til
    webløsninger/portaler.

  \item[Udvikling/vedligehold/support --] Jeg har erfaringer med
    stort set alle former for udviklingsopgaver, både sammen med
    andre og alene.

\end{description}

\subsection*{Konkrete kompetencer}

Herunder er nogle eksempler på projekter/opgaver, som jeg har været
indvolveret i:

\begin{description}

\item[Udvikling og drift af webløsning.] Færdigudvikling og
  idriftsættelse af webløsning til oprettelse,
  betaling og visning af annoncer.

  \textit{Teknologier:} Perl, Catalyst, Apache HTTP Server, CouchDB, XSLT, Linux\\
  \textit{Roller:} Udvikler, Release Manager, Support, System Administrator

\item[Migrering fra Oracle iAS til JBoss AS.] En
  plaformsmigrering af en J2EE-applikation, der var udviklet til
  Oracle iAS. Denne skulle migreres, så den kunne afviles under
  en JBoss AS. Dette indebar kodeændringer, nye deployment
  scripts, ændrede bygge-scripts samt opsætning af nye
  udviklings-, test- og produktionsmiljøer.

  \textit{Teknologier:} J2EE, Jboss, Maven, VMWare ESX\\
  \textit{Roller:} Infrastrukturarkitet, Build Master

\item[Modernisering af system til elektronisk anmeldelse.]
  Første fase bestod af tre dele: opbygning af en ny mere
  tidssvarende arkitektur, implementering af ny
  funktionalitet samt integration til legacy-systemet.

  \textit{Teknologier:} J2EE, JavaServer Faces, Hibernate, SQL, Oracle
  iAS, OC4J, Maven.\\
  \textit{Roller:} Build Master, Udvikler

\item[Webbutik] til intern handel i større dansk virksomhed
  med flere fysiske lokationer.

  \textit{Teknologier:} J2EE, Velocity, SQL, Struts, Apache Tomcat\\
  \textit{Roller:} Løsningsarkitekt, udvikler, supporter

\item[Portaler] til både medarbejdere og kunder for stor
  international virksomhed.

  \textit{Teknologier:} J2EE, SAP Enterprise Portal 6, Knowledge Manager, SQL\\
  \textit{Roller:} Udvikler, supporter

\item[Udarbejdelse af udbudsmateriale] i nært samarbejde med en
  forretningskonsulent og kunden. Dokumentation af det nuværende og
  kommende produktionssystem.

  \textit{Teknologier:} Proces-, Data Flow- og Deployment-diagrammer.\\
  \textit{Roller:} Løsningsarkitekt, forretningskonsulent

\item[Et specialudviklet CMS], der, med minimal
  administration, kan præsentere et online kursuskatalog,
  hvorfra der kan bestilles og købes kurser.

  \textit{Teknologier:} J2EE, Apache Tomcat, SQL, Perl\\
  \textit{Roller:} Udvikler, supporter

\item[Intranet og internet] for flere offentlige kunder.

  \textit{Teknologier:} ASP, Index Server, IIS\\
  \textit{Roller:} Løsningsarkitekt, udvikler

\item[Websted] i forbindelse med større tilbud.

  \textit{Teknologier:} PHP, MySQL, Apache, Fundanemt\\
  \textit{Roller:} Løsningsarkitekt, udvikler, grafiker

\item[Forretningsudvikling] i en
  tværorganisatorisk gruppe over en periode på knap et
  år.

  \textit{Rolle:} Forretningsudvikler.

\end{description}

\subsection*{Tekniske kompetencer}

Se separat bilag.

\subsection*{Sprogkundskaber}

Jeg taler og skriver engelsk. Jeg kan forstå og gøre mig forståelig på
svensk. Jeg ville gerne opfriske mit tyske, men lige nu behersker jeg
kun at læse og til dels forstå talt tysk, hvis det ikke er for
kompliceret.


\section*{Personlige oplysninger}

Jeg er 50 år, ikke-ryger.

Jeg er gift med Linda, som jeg har boet sammen med i 20 år. Vi har
ingen børn.

Linda og jeg holder meget af at se en god film, både i biografen og
derhjemme foran tv’et. Vi følger også med i (måske lidt for mange)
tv-serier, blandt favoritterne er Game of Thrones, Homeland, The
Walking Dead, American Horror Story og Criminal Minds.

Jeg spiller jævnligt spil. Ikke så meget computerspil, selvom det
hænder, men brætspil og rollespil. Det har jeg gjort lige siden
slutningen 80’erne, endda med mange af de samme venner som i dag.

Jeg er ret ferm i et køkken, og så kan jeg godt lide at lave mad. Så
både til hverdag og fest er det oftest mig, der laver mad. Til maden
drikker jeg helst en god øl, ofte udenlandsk eller fra et af de små
danske bryggerier. Interessen for god øl har jeg haft i mere end 15
år.

\end{document}
}}

\section*{Tidligere og nuværende jobs}

\textbf{AWS Solution Architect} \hfill 2018 -- (nu) \\
\textsl{KeyCore P/S}

Jeg har arbejdet som konsulent for kunder, der benytter eller har
tænkt sig at benytte Amazone Web Services (AWS).

\smallskip

\textbf{Freelancekonsulent} \hfill 2012 -- 2018 \\
\textsl{369 Consult}

\smallskip

\textbf{Seniorkonsulent, ekstern} \hfill 2014 -- (nu) \\
\textsl{Nordea A/S}

Jeg har været med til udvikling, fejlretning og 3rd level support på
en moderne webløsning til blandt andet valutahandel. Løsningen er
baseret på RESTful micro services, Dropwizard, Oracle DB, Lucene,
Hibernate og AngularJS. Desuden har jeg stået for implementeringen af
automatiseret byg, test og udrulning i Jenkins og Bamboo.

Webløsningen afløser et tidligere produkt baseret på Java Swing, Java
Web Start, Oracle DB og med flere integrationer til bagvedliggende
systemer. Jeg har også hjulpet med videreudvikling, fejlretning og 3rd
level support af dette produkt.

\smallskip

\textbf{Seniorkonsulent, ekstern} \hfill 2012 -- 2013 \\
\textsl{Visma Consulting A/S}

Fortsat som konsulent hos Visma Consulting, nu som selvstændig, se
nedenfor.

\smallskip

\textbf{Seniorkonsulent} \hfill 2011 -- 2012 \\
\textsl{Solfisk Konsulenter Aps}

Arbejdet som ekstern konsulent hos Visma Consulting, hvor jeg har
været med til at udvikle to webløsninger baseret på bl.a. JSF 2, Seam,
JBoss AS, Hibernate, XSL-FO (PDF), Web Services og Oracle
DB. Udviklingen har været styret vha.\ Scrum og jeg har arbejdet med
alle dele af udviklingen, fra back-end til front-end.  Har desuden
arbejdet med AM og videreudvikling på både disse systemer og andre
systemer, der vedligeholdes af Visma Consulting.

\smallskip

\textbf{Senior Systemudvikler} \hfill 2010 -- 2011 \\
\textsl{LeasingBørsen Aps (nu GoLease.com)}

Jeg var en del af udviklingsafdelingen, men jeg havde også et ben
driftsafdelingen. Deltog i videreudvikling og vedligehold af
forskellige egenudviklede systemer, først og fremmest en
web-applikation, der var skrevet i Perl, Catalyst og CouchDB.  Al
udvikling var styret vha.\ Scrum. Jeg var derudover, som Release
Manager, ansvarlig for integrationstest og udrulning i både test og
produktion.

\smallskip

\textbf{Seniorkonsulent} \hfill 2006 -- 2010 \\
\textsl{Sirius IT A/S (opr. TietoEnator A/S)}

Medlem af et team med fokus på én kunde og dennes webbaserede
systemer. Der har været tale om både projekter med nyudvikling samt
support og fejlretning på de eksisterende systemer. Min rolle har
primært været teknisk funderet og indadvendt mod teamet, først og
fremmest som Build Master og dermed med det tekniske ansvar for
leverancerne, og dernæst som udvikler og supporter.

\smallskip

\textbf{Senior Systemudvikler} \hfill 2000 -- 2006 \\
\textsl{CSC Consulting Group (opr. eHuset)}

Involveret i kundeprojekter af varierende størrelse. Dette har primært
været webløsninger, oftes med en bagvedliggende database og/eller
integration til eksisterende systemer. Min rolle har varieret en del,
da jeg både har arbejdet med implementering, design, behovsafdækning
(hos/med kunder), støtte for salg, support og løsningsarkitekt.

\smallskip

\textbf{Softwareudvikler} \hfill 1996 -- 2000 \\
\textsl{Dansk Maritimt Institut (nu Force Technology)}

Udvikling og vedligehold af in-house simulatorsystem og
produktificering af samme. Mine opgaver var primært udvikling, på
flere platforme og med flere teknologier – alt fra små scripts, over
biblioteker/moduler til komplette brugergrænseflader.


\section*{Uddannelse}

\textbf{Civilingeniør} \hfill 1987 -- 1995 \\
\textsl{Danmarks Tekniske Universitet}

Mit afgangsspeciale hed \textit{Rent parallaktisk stereo}, og blev
skrevet på Instituttet for matematisk modellering (IMM).


\section*{Efteruddannelse \& kurser}

\textbf{Nyheder i Java EE 6} \hfill (1 dag) 2010 \\
\textsl{Lund\&Bendsen}

\textbf{Introduktion til ITIL} \hfill (1 dag) 2008 \\
\textsl{(Internt kursus h/Sirius IT -- ekstern underviser)}

\textbf{Introduktion PRINCE2} \hfill (1 dag) 2008 \\
\textsl{(Internt kursus h/Sirius IT -- ekstern underviser)}

\textbf{BEA kursus WLP-D11-70-01-CW-2} \hfill (5 dage) 2003 \\
\textsl{Development with BEA WebLogic Portal}

\textbf{BEA selvstudie WLS-D11-61-01-CW} \hfill (2 dage) 2003 \\
\textsl{Fundamentals of J2EE Web Application Development\\WebLogic Server 6.1}

%%\textbf{Skriv bedre} \hfill (1 dag) 2000 \\
%%\textsl{IDA}
%%
%%\textbf{Præsentationsteknik} \hfill (1 dag) 2002 \\
%%\textsl{IDA}

\textbf{Databaser og Internet (ASP, SQL)} \hfill (5 dage) 1999 \\
\textsl{Dieu}

\textbf{OOD med UML notation} \hfill (3 dage) 1998 \\
\textsl{Objektorienteret Design og C++ / Java Implementering\\Teknologisk Institut}


\section*{Kompetencer}

\subsection*{Overordnede kompetencer}

\begin{description}

  \item[Kvalitetssikring/koordinering --] Jeg har arbejdet en del
    med konfigurationsstyring, automatiserede tests og Release
    Management.

  \item[Løsningsarkitekt --] Jeg er generalist, med en bred faglig
    baggrund, og får nemt overblikket over komplicerede systemer.

  \item[Tilbudsgivning/estimering --] Jeg har været med til at
    udarbejde flere, både små og store, tilbud. Desuden har jeg både
    alene og med forretningskonsulenter være med til at skrive
    anbefalinger/udbudsmateriale for kunder.

  \item[Forretningsforståelse --] Jeg er vant til kundekontakt og
    støtter ofte i forbindelse med pre-sales.

  \item[Systemintegration --] Jeg har bred teknisk viden, som jeg
    blandt andet har benyttet til integration af legacy-systemer til
    webløsninger/portaler.

  \item[Udvikling/vedligehold/support --] Jeg har erfaringer med
    stort set alle former for udviklingsopgaver, både sammen med
    andre og alene.

\end{description}

\subsection*{Konkrete kompetencer}

Herunder er nogle eksempler på projekter/opgaver, som jeg har været
indvolveret i:

\begin{description}

\item[Udvikling og drift af webløsning.] Færdigudvikling og
  idriftsættelse af webløsning til oprettelse,
  betaling og visning af annoncer.

  \textit{Teknologier:} Perl, Catalyst, Apache HTTP Server, CouchDB, XSLT, Linux\\
  \textit{Roller:} Udvikler, Release Manager, Support, System Administrator

\item[Migrering fra Oracle iAS til JBoss AS.] En
  plaformsmigrering af en J2EE-applikation, der var udviklet til
  Oracle iAS. Denne skulle migreres, så den kunne afviles under
  en JBoss AS. Dette indebar kodeændringer, nye deployment
  scripts, ændrede bygge-scripts samt opsætning af nye
  udviklings-, test- og produktionsmiljøer.

  \textit{Teknologier:} J2EE, Jboss, Maven, VMWare ESX\\
  \textit{Roller:} Infrastrukturarkitet, Build Master

\item[Modernisering af system til elektronisk anmeldelse.]
  Første fase bestod af tre dele: opbygning af en ny mere
  tidssvarende arkitektur, implementering af ny
  funktionalitet samt integration til legacy-systemet.

  \textit{Teknologier:} J2EE, JavaServer Faces, Hibernate, SQL, Oracle
  iAS, OC4J, Maven.\\
  \textit{Roller:} Build Master, Udvikler

\item[Webbutik] til intern handel i større dansk virksomhed
  med flere fysiske lokationer.

  \textit{Teknologier:} J2EE, Velocity, SQL, Struts, Apache Tomcat\\
  \textit{Roller:} Løsningsarkitekt, udvikler, supporter

\item[Portaler] til både medarbejdere og kunder for stor
  international virksomhed.

  \textit{Teknologier:} J2EE, SAP Enterprise Portal 6, Knowledge Manager, SQL\\
  \textit{Roller:} Udvikler, supporter

\item[Udarbejdelse af udbudsmateriale] i nært samarbejde med en
  forretningskonsulent og kunden. Dokumentation af det nuværende og
  kommende produktionssystem.

  \textit{Teknologier:} Proces-, Data Flow- og Deployment-diagrammer.\\
  \textit{Roller:} Løsningsarkitekt, forretningskonsulent

\item[Et specialudviklet CMS], der, med minimal
  administration, kan præsentere et online kursuskatalog,
  hvorfra der kan bestilles og købes kurser.

  \textit{Teknologier:} J2EE, Apache Tomcat, SQL, Perl\\
  \textit{Roller:} Udvikler, supporter

\item[Intranet og internet] for flere offentlige kunder.

  \textit{Teknologier:} ASP, Index Server, IIS\\
  \textit{Roller:} Løsningsarkitekt, udvikler

\item[Websted] i forbindelse med større tilbud.

  \textit{Teknologier:} PHP, MySQL, Apache, Fundanemt\\
  \textit{Roller:} Løsningsarkitekt, udvikler, grafiker

\item[Forretningsudvikling] i en
  tværorganisatorisk gruppe over en periode på knap et
  år.

  \textit{Rolle:} Forretningsudvikler.

\end{description}

\subsection*{Tekniske kompetencer}

Se separat bilag.

\subsection*{Sprogkundskaber}

Jeg taler og skriver engelsk. Jeg kan forstå og gøre mig forståelig på
svensk. Jeg ville gerne opfriske mit tyske, men lige nu behersker jeg
kun at læse og til dels forstå talt tysk, hvis det ikke er for
kompliceret.


\section*{Personlige oplysninger}

Jeg er 50 år, ikke-ryger.

Jeg er gift med Linda, som jeg har boet sammen med i 20 år. Vi har
ingen børn.

Linda og jeg holder meget af at se en god film, både i biografen og
derhjemme foran tv’et. Vi følger også med i (måske lidt for mange)
tv-serier, blandt favoritterne er Game of Thrones, Homeland, The
Walking Dead, American Horror Story og Criminal Minds.

Jeg spiller jævnligt spil. Ikke så meget computerspil, selvom det
hænder, men brætspil og rollespil. Det har jeg gjort lige siden
slutningen 80’erne, endda med mange af de samme venner som i dag.

Jeg er ret ferm i et køkken, og så kan jeg godt lide at lave mad. Så
både til hverdag og fest er det oftest mig, der laver mad. Til maden
drikker jeg helst en god øl, ofte udenlandsk eller fra et af de små
danske bryggerier. Interessen for god øl har jeg haft i mere end 15
år.

\end{document}
}}

\section*{Tidligere og nuværende jobs}

\textbf{AWS Solution Architect} \hfill 2018 -- (nu) \\
\textsl{KeyCore P/S}

Jeg har arbejdet som konsulent for kunder, der benytter eller har
tænkt sig at benytte Amazone Web Services (AWS).

\smallskip

\textbf{Freelancekonsulent} \hfill 2012 -- 2018 \\
\textsl{369 Consult}

\smallskip

\textbf{Seniorkonsulent, ekstern} \hfill 2014 -- (nu) \\
\textsl{Nordea A/S}

Jeg har været med til udvikling, fejlretning og 3rd level support på
en moderne webløsning til blandt andet valutahandel. Løsningen er
baseret på RESTful micro services, Dropwizard, Oracle DB, Lucene,
Hibernate og AngularJS. Desuden har jeg stået for implementeringen af
automatiseret byg, test og udrulning i Jenkins og Bamboo.

Webløsningen afløser et tidligere produkt baseret på Java Swing, Java
Web Start, Oracle DB og med flere integrationer til bagvedliggende
systemer. Jeg har også hjulpet med videreudvikling, fejlretning og 3rd
level support af dette produkt.

\smallskip

\textbf{Seniorkonsulent, ekstern} \hfill 2012 -- 2013 \\
\textsl{Visma Consulting A/S}

Fortsat som konsulent hos Visma Consulting, nu som selvstændig, se
nedenfor.

\smallskip

\textbf{Seniorkonsulent} \hfill 2011 -- 2012 \\
\textsl{Solfisk Konsulenter Aps}

Arbejdet som ekstern konsulent hos Visma Consulting, hvor jeg har
været med til at udvikle to webløsninger baseret på bl.a. JSF 2, Seam,
JBoss AS, Hibernate, XSL-FO (PDF), Web Services og Oracle
DB. Udviklingen har været styret vha.\ Scrum og jeg har arbejdet med
alle dele af udviklingen, fra back-end til front-end.  Har desuden
arbejdet med AM og videreudvikling på både disse systemer og andre
systemer, der vedligeholdes af Visma Consulting.

\smallskip

\textbf{Senior Systemudvikler} \hfill 2010 -- 2011 \\
\textsl{LeasingBørsen Aps (nu GoLease.com)}

Jeg var en del af udviklingsafdelingen, men jeg havde også et ben
driftsafdelingen. Deltog i videreudvikling og vedligehold af
forskellige egenudviklede systemer, først og fremmest en
web-applikation, der var skrevet i Perl, Catalyst og CouchDB.  Al
udvikling var styret vha.\ Scrum. Jeg var derudover, som Release
Manager, ansvarlig for integrationstest og udrulning i både test og
produktion.

\smallskip

\textbf{Seniorkonsulent} \hfill 2006 -- 2010 \\
\textsl{Sirius IT A/S (opr. TietoEnator A/S)}

Medlem af et team med fokus på én kunde og dennes webbaserede
systemer. Der har været tale om både projekter med nyudvikling samt
support og fejlretning på de eksisterende systemer. Min rolle har
primært været teknisk funderet og indadvendt mod teamet, først og
fremmest som Build Master og dermed med det tekniske ansvar for
leverancerne, og dernæst som udvikler og supporter.

\smallskip

\textbf{Senior Systemudvikler} \hfill 2000 -- 2006 \\
\textsl{CSC Consulting Group (opr. eHuset)}

Involveret i kundeprojekter af varierende størrelse. Dette har primært
været webløsninger, oftes med en bagvedliggende database og/eller
integration til eksisterende systemer. Min rolle har varieret en del,
da jeg både har arbejdet med implementering, design, behovsafdækning
(hos/med kunder), støtte for salg, support og løsningsarkitekt.

\smallskip

\textbf{Softwareudvikler} \hfill 1996 -- 2000 \\
\textsl{Dansk Maritimt Institut (nu Force Technology)}

Udvikling og vedligehold af in-house simulatorsystem og
produktificering af samme. Mine opgaver var primært udvikling, på
flere platforme og med flere teknologier – alt fra små scripts, over
biblioteker/moduler til komplette brugergrænseflader.


\section*{Uddannelse}

\textbf{Civilingeniør} \hfill 1987 -- 1995 \\
\textsl{Danmarks Tekniske Universitet}

Mit afgangsspeciale hed \textit{Rent parallaktisk stereo}, og blev
skrevet på Instituttet for matematisk modellering (IMM).


\section*{Efteruddannelse \& kurser}

\textbf{Nyheder i Java EE 6} \hfill (1 dag) 2010 \\
\textsl{Lund\&Bendsen}

\textbf{Introduktion til ITIL} \hfill (1 dag) 2008 \\
\textsl{(Internt kursus h/Sirius IT -- ekstern underviser)}

\textbf{Introduktion PRINCE2} \hfill (1 dag) 2008 \\
\textsl{(Internt kursus h/Sirius IT -- ekstern underviser)}

\textbf{BEA kursus WLP-D11-70-01-CW-2} \hfill (5 dage) 2003 \\
\textsl{Development with BEA WebLogic Portal}

\textbf{BEA selvstudie WLS-D11-61-01-CW} \hfill (2 dage) 2003 \\
\textsl{Fundamentals of J2EE Web Application Development\\WebLogic Server 6.1}

%%\textbf{Skriv bedre} \hfill (1 dag) 2000 \\
%%\textsl{IDA}
%%
%%\textbf{Præsentationsteknik} \hfill (1 dag) 2002 \\
%%\textsl{IDA}

\textbf{Databaser og Internet (ASP, SQL)} \hfill (5 dage) 1999 \\
\textsl{Dieu}

\textbf{OOD med UML notation} \hfill (3 dage) 1998 \\
\textsl{Objektorienteret Design og C++ / Java Implementering\\Teknologisk Institut}


\section*{Kompetencer}

\subsection*{Overordnede kompetencer}

\begin{description}

  \item[Kvalitetssikring/koordinering --] Jeg har arbejdet en del
    med konfigurationsstyring, automatiserede tests og Release
    Management.

  \item[Løsningsarkitekt --] Jeg er generalist, med en bred faglig
    baggrund, og får nemt overblikket over komplicerede systemer.

  \item[Tilbudsgivning/estimering --] Jeg har været med til at
    udarbejde flere, både små og store, tilbud. Desuden har jeg både
    alene og med forretningskonsulenter være med til at skrive
    anbefalinger/udbudsmateriale for kunder.

  \item[Forretningsforståelse --] Jeg er vant til kundekontakt og
    støtter ofte i forbindelse med pre-sales.

  \item[Systemintegration --] Jeg har bred teknisk viden, som jeg
    blandt andet har benyttet til integration af legacy-systemer til
    webløsninger/portaler.

  \item[Udvikling/vedligehold/support --] Jeg har erfaringer med
    stort set alle former for udviklingsopgaver, både sammen med
    andre og alene.

\end{description}

\subsection*{Konkrete kompetencer}

Herunder er nogle eksempler på projekter/opgaver, som jeg har været
indvolveret i:

\begin{description}

\item[Udvikling og drift af webløsning.] Færdigudvikling og
  idriftsættelse af webløsning til oprettelse,
  betaling og visning af annoncer.

  \textit{Teknologier:} Perl, Catalyst, Apache HTTP Server, CouchDB, XSLT, Linux\\
  \textit{Roller:} Udvikler, Release Manager, Support, System Administrator

\item[Migrering fra Oracle iAS til JBoss AS.] En
  plaformsmigrering af en J2EE-applikation, der var udviklet til
  Oracle iAS. Denne skulle migreres, så den kunne afviles under
  en JBoss AS. Dette indebar kodeændringer, nye deployment
  scripts, ændrede bygge-scripts samt opsætning af nye
  udviklings-, test- og produktionsmiljøer.

  \textit{Teknologier:} J2EE, Jboss, Maven, VMWare ESX\\
  \textit{Roller:} Infrastrukturarkitet, Build Master

\item[Modernisering af system til elektronisk anmeldelse.]
  Første fase bestod af tre dele: opbygning af en ny mere
  tidssvarende arkitektur, implementering af ny
  funktionalitet samt integration til legacy-systemet.

  \textit{Teknologier:} J2EE, JavaServer Faces, Hibernate, SQL, Oracle
  iAS, OC4J, Maven.\\
  \textit{Roller:} Build Master, Udvikler

\item[Webbutik] til intern handel i større dansk virksomhed
  med flere fysiske lokationer.

  \textit{Teknologier:} J2EE, Velocity, SQL, Struts, Apache Tomcat\\
  \textit{Roller:} Løsningsarkitekt, udvikler, supporter

\item[Portaler] til både medarbejdere og kunder for stor
  international virksomhed.

  \textit{Teknologier:} J2EE, SAP Enterprise Portal 6, Knowledge Manager, SQL\\
  \textit{Roller:} Udvikler, supporter

\item[Udarbejdelse af udbudsmateriale] i nært samarbejde med en
  forretningskonsulent og kunden. Dokumentation af det nuværende og
  kommende produktionssystem.

  \textit{Teknologier:} Proces-, Data Flow- og Deployment-diagrammer.\\
  \textit{Roller:} Løsningsarkitekt, forretningskonsulent

\item[Et specialudviklet CMS], der, med minimal
  administration, kan præsentere et online kursuskatalog,
  hvorfra der kan bestilles og købes kurser.

  \textit{Teknologier:} J2EE, Apache Tomcat, SQL, Perl\\
  \textit{Roller:} Udvikler, supporter

\item[Intranet og internet] for flere offentlige kunder.

  \textit{Teknologier:} ASP, Index Server, IIS\\
  \textit{Roller:} Løsningsarkitekt, udvikler

\item[Websted] i forbindelse med større tilbud.

  \textit{Teknologier:} PHP, MySQL, Apache, Fundanemt\\
  \textit{Roller:} Løsningsarkitekt, udvikler, grafiker

\item[Forretningsudvikling] i en
  tværorganisatorisk gruppe over en periode på knap et
  år.

  \textit{Rolle:} Forretningsudvikler.

\end{description}

\subsection*{Tekniske kompetencer}

Se separat bilag.

\subsection*{Sprogkundskaber}

Jeg taler og skriver engelsk. Jeg kan forstå og gøre mig forståelig på
svensk. Jeg ville gerne opfriske mit tyske, men lige nu behersker jeg
kun at læse og til dels forstå talt tysk, hvis det ikke er for
kompliceret.


\section*{Personlige oplysninger}

Jeg er 50 år, ikke-ryger.

Jeg er gift med Linda, som jeg har boet sammen med i 20 år. Vi har
ingen børn.

Linda og jeg holder meget af at se en god film, både i biografen og
derhjemme foran tv’et. Vi følger også med i (måske lidt for mange)
tv-serier, blandt favoritterne er Game of Thrones, Homeland, The
Walking Dead, American Horror Story og Criminal Minds.

Jeg spiller jævnligt spil. Ikke så meget computerspil, selvom det
hænder, men brætspil og rollespil. Det har jeg gjort lige siden
slutningen 80’erne, endda med mange af de samme venner som i dag.

Jeg er ret ferm i et køkken, og så kan jeg godt lide at lave mad. Så
både til hverdag og fest er det oftest mig, der laver mad. Til maden
drikker jeg helst en god øl, ofte udenlandsk eller fra et af de små
danske bryggerier. Interessen for god øl har jeg haft i mere end 15
år.

\end{document}
}}

\section*{Tidligere og nuværende jobs}

\textbf{AWS Solution Architect} \hfill 2018 -- (nu) \\
\textsl{KeyCore P/S}

Jeg har arbejdet som konsulent for flere kunder, hvor jeg har

\begin{itemize}
  \item implementeret flere automatiserede byggesystemer, nogle med
    \textit{cross-account} og \textit{cross-region} udrulning, baseret
    på \textit{AWS CloudFormation}, \textit{AWS CodeBuild},
    \textit{AWS CodePipeline}, \textit{CodeDeploy} and \textit{Amazon
      S3}
  \item opsat et komplet sæt \textit{Amazon ECS/ECR}-miljøer med et
    \textit{cross-account} byggesystem til at bygge et \textit{Docker
      Image} og udrulle det i hhv. udviklings-, test-, præprodutions-
    og produktionsmiljøer
  \item udviklet en \textit{AWS Lambda}, der kan styre \textit{ECR
    tasks}, for at understøtte udrulning uden nedetid
  \item udført stres- og belastningstest på midlertidige miljøer
    oprettet ved hjælp af \textit{AWS CloudFormation}
  \item arbjedet med flere andre \textit{AWS}-teknologier og -services
    som \textit{Amazon EC2}, \textit{Amazon Simple Queue Service
      (SQS)}, \textit{Amazon Simple Notification Service (SNS)},
    \textit{Amazon CloudWatch}, \textit{Amazon Relational Database
      Service (RDS)}, \textit{AWS Identity and Access Management
      (IAM)}, \textit{Amazon CloudFront}, \textit{Amazon Route 53},
    \textit{Amazon DynamoDB}, \textit{Virtual Private Cloud (VPC)},
    and \textit{AWS Database Migration Service}
\end{itemize}

\smallskip

\textbf{Seniorkonsulent, ekstern} \hfill 2014 -- (nu) \\
\textsl{Nordea A/S}

Jeg har været med til udvikling, fejlretning og 3rd level support på
en moderne webløsning til blandt andet valutahandel. Løsningen er
baseret på RESTful micro services, Dropwizard, Oracle DB, Lucene,
Hibernate og AngularJS. Desuden har jeg stået for implementeringen af
automatiseret byg, test og udrulning i Jenkins og Bamboo.

Webløsningen afløser et tidligere produkt baseret på Java Swing, Java
Web Start, Oracle DB og med flere integrationer til bagvedliggende
systemer. Jeg har også hjulpet med videreudvikling, fejlretning og 3rd
level support af dette produkt.

\smallskip

\textbf{Seniorkonsulent, ekstern} \hfill 2012 -- 2013 \\
\textsl{Visma Consulting A/S}

Fortsat som konsulent hos Visma Consulting, nu som selvstændig, se
nedenfor.

\smallskip

\textbf{Freelancekonsulent} \hfill 2012 -- 2018 \\
\textsl{369 Consult}

Startede som selvstændig freelancekonsulent.

Efter ansættelse hos KeyCore er mit freelancearbejde sat på pause.


\smallskip

\textbf{Seniorkonsulent} \hfill 2011 -- 2012 \\
\textsl{Solfisk Konsulenter Aps}

Arbejdet som ekstern konsulent hos Visma Consulting, hvor jeg har
været med til at udvikle to webløsninger baseret på bl.a. JSF 2, Seam,
JBoss AS, Hibernate, XSL-FO (PDF), Web Services og Oracle
DB. Udviklingen har været styret vha.\ Scrum og jeg har arbejdet med
alle dele af udviklingen, fra back-end til front-end.  Har desuden
arbejdet med AM og videreudvikling på både disse systemer og andre
systemer, der vedligeholdes af Visma Consulting.

\smallskip

\textbf{Senior Systemudvikler} \hfill 2010 -- 2011 \\
\textsl{LeasingBørsen Aps (nu GoLease.com)}

Jeg var en del af udviklingsafdelingen, men jeg havde også et ben
driftsafdelingen. Deltog i videreudvikling og vedligehold af
forskellige egenudviklede systemer, først og fremmest en
web-applikation, der var skrevet i Perl, Catalyst og CouchDB.  Al
udvikling var styret vha.\ Scrum. Jeg var derudover, som Release
Manager, ansvarlig for integrationstest og udrulning i både test og
produktion.

\smallskip

\textbf{Seniorkonsulent} \hfill 2006 -- 2010 \\
\textsl{Sirius IT A/S (opr. TietoEnator A/S)}

Medlem af et team med fokus på én kunde og dennes webbaserede
systemer. Der har været tale om både projekter med nyudvikling samt
support og fejlretning på de eksisterende systemer. Min rolle har
primært været teknisk funderet og indadvendt mod teamet, først og
fremmest som Build Master og dermed med det tekniske ansvar for
leverancerne, og dernæst som udvikler og supporter.

\smallskip

\textbf{Senior Systemudvikler} \hfill 2000 -- 2006 \\
\textsl{CSC Consulting Group (opr. eHuset)}

Involveret i kundeprojekter af varierende størrelse. Dette har primært
været webløsninger, oftes med en bagvedliggende database og/eller
integration til eksisterende systemer. Min rolle har varieret en del,
da jeg både har arbejdet med implementering, design, behovsafdækning
(hos/med kunder), støtte for salg, support og løsningsarkitekt.

\smallskip

\textbf{Softwareudvikler} \hfill 1996 -- 2000 \\
\textsl{Dansk Maritimt Institut (nu Force Technology)}

Udvikling og vedligehold af in-house simulatorsystem og
produktificering af samme. Mine opgaver var primært udvikling, på
flere platforme og med flere teknologier – alt fra små scripts, over
biblioteker/moduler til komplette brugergrænseflader.


\section*{Uddannelse}

\textbf{Civilingeniør} \hfill 1987 -- 1995 \\
\textsl{Danmarks Tekniske Universitet}

Mit afgangsspeciale hed \textit{Rent parallaktisk stereo}, og blev
skrevet på Instituttet for matematisk modellering (IMM).


\section*{Efteruddannelse, kurser \& certificeringer}

\textbf{AWS Certified Solutions Architect -- Associate} \hfill 2019 \\
\textsl{Amazon Web Services}

\textbf{Architecting on AWS} \hfill (3 dage) 2018 \\
\textsl{Nordcloud Training}

\textbf{Nyheder i Java EE 6} \hfill (1 dag) 2010 \\
\textsl{Lund\&Bendsen}

\textbf{Introduktion til ITIL} \hfill (1 dag) 2008 \\
\textsl{(Internt kursus h/Sirius IT -- ekstern underviser)}

\textbf{Introduktion PRINCE2} \hfill (1 dag) 2008 \\
\textsl{(Internt kursus h/Sirius IT -- ekstern underviser)}

\textbf{BEA kursus WLP-D11-70-01-CW-2} \hfill (5 dage) 2003 \\
\textsl{Development with BEA WebLogic Portal}

\textbf{BEA selvstudie WLS-D11-61-01-CW} \hfill (2 dage) 2003 \\
\textsl{Fundamentals of J2EE Web Application Development\\WebLogic Server 6.1}

%%\textbf{Skriv bedre} \hfill (1 dag) 2000 \\
%%\textsl{IDA}
%%
%%\textbf{Præsentationsteknik} \hfill (1 dag) 2002 \\
%%\textsl{IDA}

\textbf{Databaser og Internet (ASP, SQL)} \hfill (5 dage) 1999 \\
\textsl{Dieu}

\textbf{OOD med UML notation} \hfill (3 dage) 1998 \\
\textsl{Objektorienteret Design og C++ / Java Implementering\\Teknologisk Institut}


\section*{Kompetencer}

\subsection*{Overordnede kompetencer}

\begin{description}

  \item[Kvalitetssikring/koordinering --] Jeg har arbejdet en del
    med konfigurationsstyring, automatiserede tests og Release
    Management.

  \item[Løsningsarkitekt --] Jeg er generalist, med en bred faglig
    baggrund, og får nemt overblikket over komplicerede systemer.

  \item[Tilbudsgivning/estimering --] Jeg har været med til at
    udarbejde flere, både små og store, tilbud. Desuden har jeg både
    alene og med forretningskonsulenter være med til at skrive
    anbefalinger/udbudsmateriale for kunder.

  \item[Forretningsforståelse --] Jeg er vant til kundekontakt og
    støtter ofte i forbindelse med pre-sales.

  \item[Systemintegration --] Jeg har bred teknisk viden, som jeg
    blandt andet har benyttet til integration af legacy-systemer til
    webløsninger/portaler.

  \item[Udvikling/vedligehold/support --] Jeg har erfaringer med
    stort set alle former for udviklingsopgaver, både sammen med
    andre og alene.

\end{description}

\subsection*{Konkrete kompetencer}

Herunder er nogle eksempler på projekter/opgaver, som jeg har været
indvolveret i:

\begin{description}

\item[Udvikling og drift af webløsning.] Færdigudvikling og
  idriftsættelse af webløsning til oprettelse,
  betaling og visning af annoncer.

  \textit{Teknologier:} Perl, Catalyst, Apache HTTP Server, CouchDB, XSLT, Linux\\
  \textit{Roller:} Udvikler, Release Manager, Support, System Administrator

\item[Migrering fra Oracle iAS til JBoss AS.] En
  plaformsmigrering af en J2EE-applikation, der var udviklet til
  Oracle iAS. Denne skulle migreres, så den kunne afviles under
  en JBoss AS. Dette indebar kodeændringer, nye deployment
  scripts, ændrede bygge-scripts samt opsætning af nye
  udviklings-, test- og produktionsmiljøer.

  \textit{Teknologier:} J2EE, Jboss, Maven, VMWare ESX\\
  \textit{Roller:} Infrastrukturarkitet, Build Master

\item[Modernisering af system til elektronisk anmeldelse.]
  Første fase bestod af tre dele: opbygning af en ny mere
  tidssvarende arkitektur, implementering af ny
  funktionalitet samt integration til legacy-systemet.

  \textit{Teknologier:} J2EE, JavaServer Faces, Hibernate, SQL, Oracle
  iAS, OC4J, Maven.\\
  \textit{Roller:} Build Master, Udvikler

\item[Webbutik] til intern handel i større dansk virksomhed
  med flere fysiske lokationer.

  \textit{Teknologier:} J2EE, Velocity, SQL, Struts, Apache Tomcat\\
  \textit{Roller:} Løsningsarkitekt, udvikler, supporter

\item[Portaler] til både medarbejdere og kunder for stor
  international virksomhed.

  \textit{Teknologier:} J2EE, SAP Enterprise Portal 6, Knowledge Manager, SQL\\
  \textit{Roller:} Udvikler, supporter

\item[Udarbejdelse af udbudsmateriale] i nært samarbejde med en
  forretningskonsulent og kunden. Dokumentation af det nuværende og
  kommende produktionssystem.

  \textit{Teknologier:} Proces-, Data Flow- og Deployment-diagrammer.\\
  \textit{Roller:} Løsningsarkitekt, forretningskonsulent

\item[Et specialudviklet CMS], der, med minimal
  administration, kan præsentere et online kursuskatalog,
  hvorfra der kan bestilles og købes kurser.

  \textit{Teknologier:} J2EE, Apache Tomcat, SQL, Perl\\
  \textit{Roller:} Udvikler, supporter

\item[Intranet og internet] for flere offentlige kunder.

  \textit{Teknologier:} ASP, Index Server, IIS\\
  \textit{Roller:} Løsningsarkitekt, udvikler

\item[Websted] i forbindelse med større tilbud.

  \textit{Teknologier:} PHP, MySQL, Apache, Fundanemt\\
  \textit{Roller:} Løsningsarkitekt, udvikler, grafiker

\item[Forretningsudvikling] i en
  tværorganisatorisk gruppe over en periode på knap et
  år.

  \textit{Rolle:} Forretningsudvikler.

\end{description}

\subsection*{Tekniske kompetencer}

Se separat bilag (der kan hentes her: \url{https://slu.sdf.org/resume/kompetencer.pdf})

\subsection*{Sprogkundskaber}

Jeg taler og skriver engelsk. Jeg kan forstå og gøre mig forståelig på
svensk. Jeg ville gerne opfriske mit tyske, men lige nu behersker jeg
kun at læse og til dels forstå talt tysk, hvis det ikke er for
kompliceret.


\section*{Personlige oplysninger}

Jeg er 52 år, ikke-ryger.

Jeg er gift med Linda, som jeg har boet sammen med i mere end 20
år. Vi har ingen børn.

Linda og jeg holder meget af at se en god film, både i biografen og
derhjemme foran tv’et. Vi følger også med i (måske lidt for mange)
tv-serier, blandt favoritterne er Game of Thrones, Homeland, The
Walking Dead, American Horror Story og Criminal Minds.

Jeg spiller jævnligt spil. Ikke så meget computerspil, selvom det
hænder, men brætspil og rollespil. Det har jeg gjort lige siden
slutningen 80’erne, endda med mange af de samme venner som i dag.

Jeg er ret ferm i et køkken, og så kan jeg godt lide at lave mad. Så
både til hverdag og fest er det oftest mig, der laver mad. Til maden
drikker jeg helst en god øl, ofte udenlandsk eller fra et af de små
danske bryggerier. Interessen for god øl har jeg haft i mere end 15
år.

\end{document}
